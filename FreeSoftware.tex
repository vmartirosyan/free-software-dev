
\input Header.tex 

\begin{document}



%-------- Empty --------

%\input Preamble/title_ru.tex
\input Preamble/title_en.tex

%\input Preamble/ISBN.tex

\newpage

\tableofcontents


\begin{center}
\chapter*{PREFACE}
\sectionmark*
\end{center}

This book is a convenient start point for those who plan to start development of free software. As the founder of the Free Software Movement is Richard Stallman, we shall refer to his definitions, concepts and philosophy along the material. Here you will find the definition of a free software, it's main characteristics, it's differences from non-free software, the ideology of freedom in software world. This will help the reader to decide if the free software is what he is searching for.

After understanding the fundamentals of free software, we'll start studying free tools made to ease the process of software development. The list of tools includes editors, IDEs, compiler, GNU make utility, automake tools, debugger. These all will make possible the realization of the last part of the book.

Of course, the development tools are really necessary for the development process, but it would still be impossible to create complicated projects without libraries. We are going to study several free libraries, developed by GNU, which cover the whole range of application development: starting with the utility libraries like GLib, GObject, and till the graphical interface, like GDK and GTK+.

At last we are going to develop a complete free system of vehicle tracking. It will include a listener server, a database, a web-service and a user application, which allows the user to see it's vehicles. Here all the code parts will be presented in C++. It is a necessary point for the reader to be familiar with the C++ language.

\let\cleardoublepage\clearpage

\input chapter1/ch1p1.tex
\input chapter1/ch1p2.tex
\input chapter1/ch1p3.tex
\input chapter1/ch1p4.tex
\input chapter1/ch1p5.tex
%\input chapter1/ch1p6.tex
%\input chapter1/ch1p7.tex
%\input chapter1/ch1p8.tex
%\input chapter1/ch1p9.tex


\input chapter2/ch2p1.tex
\input chapter2/ch2p2.tex
\input chapter2/ch2p3.tex
\input chapter2/ch2p4.tex
\input chapter2/ch2p5.tex
\input chapter2/ch2p6.tex

\input chapter3/ch3p1.tex
\input chapter3/ch3p2.tex
\input chapter3/ch3p3.tex
\input chapter3/ch3p4.tex
\input chapter3/ch3p5.tex
\input chapter3/ch3p6.tex
\input chapter3/ch3p7.tex
\input chapter3/ch3p8.tex

\input chapter4/ch4p1.tex
\input chapter4/ch4p2.tex
\input chapter4/ch4p3.tex
\input chapter4/ch4p4.tex
\input chapter4/ch4p5.tex

\newpage
%\input vyxodnye.tex

\end{document}
