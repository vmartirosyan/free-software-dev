\documentclass[unicode, 12pt, a4paper, doubleside]{book}		
% Mirrors the output
%\usepackage[mirror]{crop}
\usepackage[xetex]{hyperref}
%\usepackage[russian,english]{babel}
%\usepackage[cm-default]{fontspec}
%\defaultfontfeatures{Mapping=tex-text}
    %% устанавливает поведение шрифтов по умолчанию
%\usepackage[armenian,russian,english]{polyglossia}
    %% подключает пакет многоязыкой верстки	
%\setdefaultlanguage{russian}
    %% устанавливает язык по умолчанию
%\setmainfont{DejaVu Serif}  
%\setmainfont{Free Sans}  
	%% задаёт основной шрифт документа
\linespread{1.3}\normalsize
\usepackage{graphicx}

\usepackage{listings}
\usepackage{color}
\usepackage{xcolor}
\usepackage{courier}	
\usepackage[large]{caption}

\usepackage{textcomp}

\setlength{\textwidth}{16cm}
\setlength{\textheight}{24cm}
\setlength{\oddsidemargin}{0in}
\setlength{\evensidemargin}{0in}
\setlength{\topmargin}{0in}

\clubpenalty=10000
\widowpenalty=10000

\newfontface \armfont [Script=Armenian]{Sylfaen}
\def \ARMENIAN #1{{\armfont #1}}

\renewcommand{\subsectionmark}[1]{}

%\renewcommand{\@evenhead}%
%{\vbox{\hbox to\textwidth{\thepage \hfil \centering{\normalfont\slshape \leftmark} \hfil}\smallskip\hrule}}

%\renewcommand{\@oddhead}{\vbox{\hbox to\textwidth{\hfil \centering{\normalfont\slshape\rightmark} \hfil \thepage}\smallskip\hrule}}

%\newtheorem{definition}{Определение}[section]
%\newtheorem{theorem}{Теорема}[section]
%\newtheorem{lemma}{Лемма}[section]
%\newtheorem{remark}{Замечание}[section]
%\newtheorem{example}{Пример}[section]
%\renewcommand{\bibname}{Литература}
%\renewcommand{\chaptername}{Глава}
%\renewcommand{\contentsname}{С\,О\,Д\,Е\,Р\,Ж\,А\,Н\,И\,Е}
%\renewcommand{\baselinestretch}{1.6}

\newtheorem{definition}{Definition}[section]
\newtheorem{theorem}{Theorem}[section]
\newtheorem{lemma}{Lemma}[section]
\newtheorem{remark}{Remark}[section]
\newtheorem{example}{Example}[section]
\renewcommand{\bibname}{Literature}
\renewcommand{\chaptername}{Chapter}
\renewcommand{\contentsname}{C\,O\,N\,T\,E\,N\,T\,S}

\lstset{
         basicstyle=\normalsize\rmfamily, % Standardschrift
         numbers=left,               % Ort der Zeilennummern
         numberstyle=\normalsize,          % Stil der Zeilennummern
         %stepnumber=2,               % Abstand zwischen den Zeilennummern
         numbersep=5pt,              % Abstand der Nummern zum Text
         tabsize=2,                  % Groesse von Tabs
         %extendedchars=true,         %
         breaklines=true,            % Zeilen werden Umgebrochen
         keywordstyle=\color{blue},
		 %keywordstyle=\color{black},
		 frame=none,
%         frame=b, 
 %        keywordstyle=[1]\textbf,    % Stil der Keywords
 %        keywordstyle=[2]\textbf,    %
 %        keywordstyle=[3]\textbf,    %
 %        keywordstyle=[4]\textbf,   \sqrt{\sqrt{}} %
         stringstyle=\color{green}\rmfamily, % Farbe der String
		 %stringstyle=\color{black}\rmfamily, % Farbe der String
         showspaces=false,           % Leerzeichen anzeigen ?
         showtabs=false,             % Tabs anzeigen ?
         xleftmargin=17pt,
         framexleftmargin=17pt,
         framexrightmargin=5pt,
         framexbottommargin=4pt,
         %backgroundcolor=\color{lightgray},
         showstringspaces=false      % Leerzeichen in Strings anzeigen ?        
 }
 \lstloadlanguages{% Check Dokumentation for further languages ...
         %[Visual]Basic
         %Pascal
         C,
		 %Assembler,
         C++,
         XML
		 %makefile
         %HTML
         %Java
 }
   %\DeclareCaptionFont{gray}{\color{gray}} 
   \DeclareCaptionFont{gray}{\color{black}} 

  \captionsetup[lstlisting]{singlelinecheck=false, labelfont={gray}, textfont={gray}}
  %\captionsetup[lstlisting]{singlelinecheck=false, labelfont={black}, textfont={black}}
  
%\DeclareCaptionFont{white}{\color{white}}
\DeclareCaptionFormat{listing}{\parbox{\textwidth}{\hspace{15pt}#1#2#3}}
\captionsetup[lstlisting]{format=listing,labelfont=gray,textfont=gray, singlelinecheck=false, margin=0pt, font={bf,normalsize}}
